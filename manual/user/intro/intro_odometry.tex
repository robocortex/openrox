\chapter{Introduction}
\label{cha:intro}

\rox{} is a computer vision software written in strict ANSI-C
and optimized for real-time applications. The optimization has been performed with SSE 4.2 for x86 processors and Neon for Arm processors.
No extra dependencies are required to compile the software, just the standard C library. The software provides advanced
algorithms that can be used for applications like vision-guided robot
control and augmented reality. \\

Contrarily to standard computer vision software, \rox{} uses {\bf
direct methods} and {\bf dense information} to solve computer vision
problems. This approach is more robust and accurate than sparse
features-based methods, but generally requires high computation
resources. The use of a new fast optimization technique called {\bf
ESM (Efficient Second-order Approximation Method)} allows to use dense
direct methods in real-time. The ESM technique was proved to have a
higher convergence rate (third-order) than the Newton optimization
technique (second-order) while being more efficient.\\

The present version of the \rox{} is composed of the following modules~:

\begin{description}
  \item[Utils]~: module with basic structures and functions used by all the other modules ;
  \item[Maths]~: module with mathematical structures and methods ;
  \item[Model]~: module with structures and functions to handle target models ;
  \item[Vision]~: module with structures and functions for computer vision ;
  \item[Sensor]~: module with structures and functions for handling sensors ;
  \item[Odometry]~: module with structures and methods for visual odometry computation (sensor localization). The output of the module is the pose of the sensor (translation and rotation in the Cartesian space) relative to a target model ;

\end{description}

\section{Quick guide}
\label{sec:install}

\noindent In the directory \roxdir{}, you will find the following directories:

\begin{itemize}
  \item bin~: The directory containing the dynamic libraries and used to build the executable examples. By default, the license file shall be copied to this directory.
  \item doc~: The directory containing the documentation~;
  \item inc~: The directory containing the headers for the various structures and functions~;
  \item lib~: The directory containing the static libraries~;
  \item lic~: The directory containing the license file~;
  \item res~: Directory needed to save example results~;
  \item seq~: The directory containing the sequence of images (.pgm files) used to test the applications. More test sequences can be downloaded from Robocortex website at www.robocortex.com. Instructions are given in eache example file;
  \item src~: Directory containing several examples for using \rox{}. Examples are provided to test the library and commented in order to
help the user understand how to use and make his own application.;
\end{itemize}

\noindent To compile the examples go to the bin directory:
\begin{lstlisting}
cd bin
\end{lstlisting}
then use cmake-gui (can be downloaded from http://www.cmake.org/) to generate the platform specific projects to compile the examples: 
\begin{lstlisting}
cmake-gui ..
\end{lstlisting}
Make sure that cmake-gui is added to the system PATH. 

\section{License file}
\label{sec:license_file}

A license file shall be obtained to use \rox{} and run the example. By default, the license file shall be copied to the "lic" directory. The user can put the file in a different directory by changing the path in the example files.

%\input{intro/coding.tex}
