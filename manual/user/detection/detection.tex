\chapter{Motion Detection Module}
\label{cha:motion_detection}

Visual motion detection is the process of determining the changes
occurring in a sequence of images relative to a given reference
image. If the camera is static, these changes are mainly due to moving
objects. The localization of moving objects in the image is used in
several applications such as robotics, active video-surveillance, and many
others.

In \rox{}, the detection of a moving target is done inside an
appropriate Region Of Interest (ROI) selected by the user. In the
current version of \rox{} the image is supposed to be acquired by a
static camera.

The Detection module contains the following sub-modules:

\begin{description}
\item[Detection Parameters]~: module with structures and methods for handling the detection parameters~;
\item[Detection]~: module with structures and methods for motion detection ~;
\end{description}

\section{Detection Parameters}
\label{sec:detection_params}

\subsection{The Rox\_Detection\_Params object}
\label{sse:detection_params_object}

The object \lstinline$Rox_Detection_Params$ contains two
parameters that can be used to define the visual detection.
 The parameters are the following:

\begin{itemize}
\item Bandwidth : the approximate size of the object to be detected 
\item Sensitivity : the motion detection sensitivity
\end{itemize}

The detection parameters are stored in a \lstinline$Rox_Detection_Params_Struct$ structure. A \lstinline$Rox_Detection_Params$ object is opaque and can be accessed using the pointer to a \lstinline$Rox_Detection_Params_Struct$ structure: 
\begin{lstlisting}
typedef struct Rox_Detection_Params_Struct* Rox_Detection_Params;
\end{lstlisting}

\subsection{Creating/Deleting a Rox\_Detection\_Params object}
\label{sse:detection_params_creating}

\rox{} provides functions to create / delete the Rox\_Detection\_Params structure~:

\begin{lstlisting}
Rox_Detection_Params rox_detection_params_new(Rox_Void);
\end{lstlisting}
The \lstinline$rox_detection_params_new$ function allocates memory for the structure, 
sets parameters with default values (\lstinline$bandwidth = (20x30) pixels$, \lstinline$sensitivity = 1.0$) and returns a pointer on the newly created structure.\\

\begin{lstlisting}
Rox_Void rox_detection_params_del(Rox_Detection_Params Params);
\end{lstlisting}
The \lstinline$rox_detection_params_del$ function deallocates memory for a parameter structure. It
is necessary to call this function when the structure is not used anymore or before overwriting it.

\subsection{Main functions related to the object Rox\_Detection\_Params}
\label{sse:detection_params_functions}

The motion detection algorithm has two important parameters set at the parameters structure
initialization~: `bandwidth', `sensitivity'. Functions are provided to set these parameters~:

\begin{itemize}
  \item \lstinline$rox_detection_params_set_bandwidth$~: Sets the bandwidth parameter which is the approximate size of the moving object to detect. If the bandwidth is small, parts of bigger objects will be detected separately. On the contrary, if the   bandwidth is big, smaller objects may be grouped together. 
  \item \lstinline$rox_detection_params_set_sensitivity$~: Sets sensitivity parameter. The higher the sensitivity the more objects will be detected but false detections may appear (due to lighting changes, noise, or other minor changes in the image).
\end{itemize}

Setting a parameter with a non valid value (for example -1) will leave it unchanged. \\

Finally, the function \lstinline$rox_detection_params_set_license_path$ allows the user to set the path of the license file.


\section{Detection}
\label{sec:detetion}

\subsection{The Rox\_Detection object}
\label{sse:detection_object}

A \lstinline$Rox_Detection$ object can be defined using the pointer to a \lstinline$Rox_Detection_Structure$:
\begin{lstlisting}
typedef struct Rox_Detection_Struct* Rox_Detection;
\end{lstlisting}

\subsection{Creating/Deleting a Rox\_Detection object}
\label{sss:detection_initializing}

Functions are provided to allocate and deallocate a \lstinline$Rox_Detection$ structure~:
\begin{lstlisting}
Rox_Detection rox_detection_new(Rox_Detection_Params detection_params, Rox_Image image);
\end{lstlisting}
The \lstinline$rox_detection_new$ function allocates memory for the structure of the motion detection, according to
the \lstinline$detection_params$ object and returns a pointer on the newly created structure. The input image corresponds to the model image (background) used for the motion detection. \\

\begin{lstlisting}
Rox_Void rox_detection_del(Rox_Detection detection);
\end{lstlisting}
The \lstinline$rox_detection_del$ function deallocates memory for a \lstinline$Rox_Detection$ structure. 
It is necessary to call this function when the structure is not used anymore or before overwriting it. 
Remember to set \lstinline$detection = NULL;$ after deleting the object.
\\

The \lstinline$Rox_Detection$ structure is opaque, i.e. its internal
fields are hidden and cannot be accessed directly by the user. A
consequence is that the user can only declare and manipulate pointer
on this structure, and never the structure itself, as the size of the
structure is unknown to the user.\\

In order to detect moving objects in several ROI, the user can declare and create
several \lstinline$Rox_Detection$ structures and perform the algorithm on each of
them. \\

\subsection{The main functions related to the object Rox\_Detection}
\label{sss:detection_methods}

Once the \lstinline$Rox_Detection$ is allocated and initialized, it is possible to:
\begin{itemize}
  \item Set a different model image
  \item Set an image mask to avoid detection in specific parts of the ROI
  \item Set different parameters for each detection
  \item Detect moving objects
\end{itemize}

\subsubsection{Setting a model image}
\label{sse:detection_set_model}

The function \lstinline$rox_detection_motion_set_model$ allows the
user to set a model image which may be different from the model image
chosen at initialization. This can be useful if the appearance of the
model has changed. For example the detection may start in the morning
and in the afternoon the lighting conditions have changed so much
that it is preferable to renew the model image.\\

\subsubsection{Setting an image mask}
\label{sse:detection_set_imask}

The following functions allow the user to set an image mask of any shape. This mask can be used to avoid
detection in specific parts of the ROI. \\

\paragraph{Setting an image mask from windows}
\label{sss:detection_set_imask_windows}
~\\
The function \lstinline$rox_detection_set_imask_window$ shall be used
to select a rectangular region defined by a window.

The function \lstinline$rox_detection_motion_set_imask_window_list$ shall be used to select several rectangular regions defined by a window list.

\paragraph{Setting an user defined image mask}
\label{sss:detection_set_imask_user}
~\\

The function \lstinline$rox_detection_set_imask$ shall be used for defining such regions (for example a disc).\\
Further information about image masks can be found in section \ref{sec:imask}. Detection can then be performed with
\lstinline$rox_detection_make$ function.

\subsubsection{Setting detection parameters}
\label{sse:detection_set_params}

The motion detection parameters (see section \ref{sec:detection_params}) can be modified for each image using the following functions:

\lstinline$rox_detection_motion_set_bandwidth$

\lstinline$rox_detection_motion_set_sensitivity$

%\paragraph{Setting the size of the detected objetcs}
%\label{sss:detection_set_bandwidth}

%\paragraph{Setting the detection sensitivity}
%\label{sss:detection_set_sensitivity}

\subsubsection{Performing motion detection}
\label{sss:detection_set_params}

To perform the detection we call the \lstinline$rox_detection_make$
function with the \lstinline$Rox_Detection$ structure and the image
used to perform the detection as parameters. This function shall be
called for each image of the sequence in which the user would like to
detect a moving object.

The output of the detection is a list of windows defining a bounding
box around the moving object.

See the example ``rox\_example\_detection.tex'' for an example of use.

%\input{detection/rox_example_detection}
