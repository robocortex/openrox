\section{Inertia}
\label{sec:inertia}

The inertia module, only used for the visual - inertial odometry, is composed by two sub-modules. The {\tt Rox\_Inertial} object contains the accelerometer ($m/s^2$), gyrometer ($rad/s$) and magnetometer calibrated measures and the acquisition timestamp. This data is required by the visual - inertial odometry and available through the {\tt Rox\_Inertial} object. Indeed, after setting the {\tt Rox\_Inertial} object with valid data, the user is able to make visual - inertial odometry using the functions described in section \ref{sss:odometry_visual_inertial_methods} and the {\tt Rox\_Inertial} object.

\subsection{The {\tt Rox\_Inertial} object}
\label{sse:inertial_struct}

A \lstinline$Rox_Inertial$ object can be defined using the pointer to a \lstinline$Rox_Inertial_Structure$:
\begin{lstlisting}
typedef struct Rox_Inertial_Struct* Rox_Inertial;
\end{lstlisting}

\subsection{Creating/Deleting a {\tt Rox\_Inertial}}
\label{sse:inertial_newdel}

Functions are provided to allocate and deallocate a \lstinline$Rox_Inertial$ object~:

\begin{lstlisting}
Rox_Error rox_inertial_new (Rox_Inertial *inertial, const Rox_Float frequency);
\end{lstlisting}
The function allocates memory for the inertial object and returns a pointer on the newly created object.

\begin{lstlisting}
Rox_Error rox_inertial_del (Rox_Inertial *inertial);
\end{lstlisting}
The function deallocates memory for a \lstinline$Rox_Inertial$ object. 
It is necessary to call this function when the object is not used anymore. \\

\subsection{Main functions related to {\tt Rox\_Inertial}}
\label{sse:inertial_functs}

The inertial measures can be set in the \lstinline$Rox_Inertial$ object using the following function~:

\begin{lstlisting}
Rox_Error rox_inertial_set_measure(Rox_Inertial inertial, const Rox_Real* A, const Rox_Real* W, const Rox_Real* M, Rox_Real timestamp);
\end{lstlisting}

This function set the accelerometer, gyrometer and magnetometer measures from \lstinline$Rox_Real$ buffers and sets the acquisition timestamp~;

If you need further information about inertial functions, please refer to the Programmer Manual.

