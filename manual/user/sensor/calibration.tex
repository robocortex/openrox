\subsection{Camera Calibration}
\label{sse:camera_calibration}

\noindent Camera calibration consists in finding the intrinsic parameters of the following matrix (see section~\ref{sec:camera}):

\begin{equation}
\bf{K} = \left[ \begin{array}{ccc} f & s & c_u \\ 0 & f r & c_v \\ 0 & 0 & 1 \end{array} \right]
\end{equation}

\noindent This tedious procedure is extremely easy with \rox{}.

\noindent The user can print or display on his screen any image with sufficient texture. After measuring the size of the rectangle the user can go through the following work-flow for camera calibration: 

\begin{itemize}
\item create a new calibration object using a 2D model with known size:

\begin{lstlisting}
Rox_Error rox_texture_calibration_mono_perspective_new(Rox_Texture_Calibration_Mono_Perspective *calibration, Rox_Model_2D model);
\end{lstlisting}

The size of the printed texture is given in meters.

\item add a calibration image:
 
\begin{lstlisting}
Rox_Error rox_texture_calibration_mono_perspective_add_image(Rox_Texture_Calibration_Mono_Perspective calibration, Rox_Image image);
\end{lstlisting}

The user shall add images (up to 20) taken with the camera and viewing
the same texture from different point of views. An inclination between
30 and 45 degrees relative to the normal to the texture plane wil
give good results. For optimal results, the texture in the current
image should have almost the ssame size of the model image. The more
images are added, the more camera intrinsic parameters ca be
calibrated (see the ``method'' parameter below).

\item make the camera calibration:
 
\begin{lstlisting}
Rox_Error rox_texture_calibration_mono_perspective_make(Rox_Texture_Calibration_Mono_Perspective calibration, Rox_Uint method);
\end{lstlisting}

The user can choose the ``method'' parameter (between 1 and 5) to calibrate the following camera parameters:

\begin{enumerate}
\item method = 1 : calibrate the focal length $f$ only (assuming the
  principal point $c_u$, $c_v$ are at the center of the image). This
  method needs only 1 image in which the texture model is not observed
  fronto-parallel to the 3D plane.
\item method = 2 : calibrate the focal length $f$ and aspect ration $r$
  (assuming the principal point $c_u$, $c_v$ are at the center of the
  image). This method needs only 1 image in which the texture model is
  not observed fronto-parallel to the 3D plane.
\item method = 3 : calibrate the focal length $f$ and the principal point $c_u$, $c_v$. This method needs at least 2 images in which the texture model is
  not observed fronto-parallel to the 3D plane.
\item method = 4 : calibrate the focal length $f$, aspect ration $r$
  and the principal point $c_u$, $c_v$. This method needs at least 2
  images in which the texture model is not observed fronto-parallel to
  the 3D plane.
\item method = 5 : calibrate the focal length $f$, aspect ration $r$, skew $s$
  and the principal point $c_u$, $c_v$. This method needs at least 3
  images in which the texture model is not observed fronto-parallel to
  the 3D plane.
\end{enumerate}

Use the method number 4 when calibrating the camera for computing the odometry with \rox{}.

\item get the camera calibration:

\begin{lstlisting}
Rox_Error rox_texture_calibration_mono_perspective_get_intrinsics(Rox_Matrix intrinsics, Rox_Texture_Calibration_Mono_Perspective calibration);
\end{lstlisting}

The camera intrinsic parameters are written in a 3x3 matrix.
\end{itemize}

See the example ``rox\_example\_camera\_calibration.c'' for an example of camera calibration.
