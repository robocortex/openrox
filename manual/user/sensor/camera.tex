\section{Camera}
\label{sec:camera}
\rox assumes that images are acquired by a camera that respects a standard pin-hole model. 
With this model, straight lines in the 3D space will project to straight lines in the 2D image space.

In a pin-hole model, the relation between a point M with coordinates
(X,Y,Z,1) in reference frame and its equivalent with homogeneous
coordinates (u,v,1) in the camera frame can be written as follows~:
\begin{center}
$
\left( \begin{array}{c} u \\ v \\ 1 \\ \end{array} \right) \propto
\underbrace{\left[ \begin{array}{ccc} f & s & c_u \\ 0 & f r & c_v \\ 0 & 0 & 1 \end{array} \right]}_{\mbox{\footnotesize{K }}} 
\underbrace{\left[ \begin{array}{cccc} & & & t_x \\ & R_{3\times3} & & t_y \\ & & & t_z \\ 0&0&0&1 \end{array} \right]}_{\mbox{\footnotesize{T}}} 
\left( \begin{array}{c} X \\ Y \\ Z \\ 1\\ \end{array} \right)
$
\end{center}

The K matrix corresponds to camera intrinsic parameters where:
\begin{description}
   \item [f]~: Focal lenght in pixels
   \item [r]~: Pixel ratio (equal to 1 for square pixels)
   \item [s]~: Skew (equal to 0 for square or rectangular pixels)
   \item [$c_u$]~: u-coordinate of the optical center projection in the image
   \item [$c_v$]~: v-coordinate of the optical center projection in the image
\end{description}

The intrinsic parameters are generally fixed by the hardware and should not change with time.

The procedure to calibrate the intrinsic parameters is described in the section~\ref{sse:camera_calibration}\\

The T matrix corresponds to camera extrinsic parameters where:
\begin{description}
   \item [R]~: Rotation matrix between reference and current position of the camera
   \item [t]~: Translation vector between reference and current position of the camera
\end{description}

The extrinsic parameters change depending on the camera localization and have to be computed for each image.\\

{\bf \rox} provides functions and structures to handle these parameters. The intrinsic and extrinsic parameters are stored in \lstinline$Rox_Camera$ structure.
% The extrinsic parameters are stored in \lstinline$Rox_Frame$ structure.\\

\subsection{The {\tt Rox\_Camera} object}
\label{sse:camera_struct}

The structure \lstinline$Rox_Camera_Struct$ is designed to hold camera data: intrinsic and extrinsic parameters matrices and the image. 

A camera object is opaque and can be accessed using the pointer \lstinline$Rox_Camera$ to a \lstinline$Rox_Camera_Struct$ structure. 
\begin{lstlisting}
typedef struct Rox_Camera_Struct* Rox_Camera;
\end{lstlisting}

\subsection{Creating/Deleting a {\tt Rox\_Camera}}
\label{sse:camera_newdel}

The library provides functions to create, initialize and delete \lstinline$Rox_Camera$ structure.
\begin{lstlisting}
Rox_Error rox_camera_new (Rox_Camera *camera, Rox_Uint cols, Rox_Uint rows);
\end{lstlisting}
The \lstinline$rox_camera_new$ function allocates memory for data. In this case, the size of allocated memory depends on parameters `cols' and `rows' which correspond to image size. 

% The intrinsic and extrinsic parameters are respectively initialized with `K' and `T'.\\

% If the parameters are unknown, `K' and `T' can be set to NULL.

The \lstinline$rox_camera_new_readpgm$ function allocates memory for data read from a pgm file (`filename'):
\begin{lstlisting}
Rox_Error rox_camera_new_readpgm(Rox_Camera * camera, Rox_Char* filename);
\end{lstlisting}
If the user has a different file format (for example png or jpeg files), he can use its own library to load the image and then fill the Rox\_Image object contained in the Rox\_Camera object (see the Image module).

The \lstinline$rox_camera_del$ function deallocates memory for an \lstinline$Rox_Camera$ structure. It is necessary to call this function when the structure is not used anymore:
\begin{lstlisting}
Rox_Error rox_camera_del(Rox_Camera * camera);
\end{lstlisting}

\subsection{Main functions related to {\tt Rox\_Camera}}
\label{sse:camera_functs}

The main functions to use an \lstinline$Rox_Camera$ structure are~:
\begin{description}
  \item [rox\_camera\_readpgm]~: Sets image field by loading a pgm file.
  \item [rox\_camera\_get\_pose]~: Returns the pointer to the pose matrix.
% \item [rox\_camera\_get\_frame]~: Returns the frame structure.
  \item [rox\_camera\_set\_image]~: Set the image in the camera.
\end{description}
 
\subsection{Camera Calibration}
\label{sse:camera_calibration}

\noindent Camera calibration consists in finding the intrinsic parameters of the following matrix (see section~\ref{sec:camera}):

\begin{equation}
\bf{K} = \left[ \begin{array}{ccc} f & s & c_u \\ 0 & f r & c_v \\ 0 & 0 & 1 \end{array} \right]
\end{equation}

\noindent This tedious procedure is extremely easy with \rox{}.

\noindent The user can print or display on his screen any image with sufficient texture. After measuring the size of the rectangle the user can go through the following work-flow for camera calibration: 

\begin{itemize}
\item create a new calibration object using a 2D model with known size:

\begin{lstlisting}
Rox_Error rox_texture_calibration_mono_perspective_new(Rox_Texture_Calibration_Mono_Perspective *calibration, Rox_Model_2D model);
\end{lstlisting}

The size of the printed texture is given in meters.

\item add a calibration image:
 
\begin{lstlisting}
Rox_Error rox_texture_calibration_mono_perspective_add_image(Rox_Texture_Calibration_Mono_Perspective calibration, Rox_Image image);
\end{lstlisting}

The user shall add images (up to 20) taken with the camera and viewing
the same texture from different point of views. An inclination between
30 and 45 degrees relative to the normal to the texture plane wil
give good results. For optimal results, the texture in the current
image should have almost the ssame size of the model image. The more
images are added, the more camera intrinsic parameters ca be
calibrated (see the ``method'' parameter below).

\item make the camera calibration:
 
\begin{lstlisting}
Rox_Error rox_texture_calibration_mono_perspective_make(Rox_Texture_Calibration_Mono_Perspective calibration, Rox_Uint method);
\end{lstlisting}

The user can choose the ``method'' parameter (between 1 and 5) to calibrate the following camera parameters:

\begin{enumerate}
\item method = 1 : calibrate the focal length $f$ only (assuming the
  principal point $c_u$, $c_v$ are at the center of the image). This
  method needs only 1 image in which the texture model is not observed
  fronto-parallel to the 3D plane.
\item method = 2 : calibrate the focal length $f$ and aspect ration $r$
  (assuming the principal point $c_u$, $c_v$ are at the center of the
  image). This method needs only 1 image in which the texture model is
  not observed fronto-parallel to the 3D plane.
\item method = 3 : calibrate the focal length $f$ and the principal point $c_u$, $c_v$. This method needs at least 2 images in which the texture model is
  not observed fronto-parallel to the 3D plane.
\item method = 4 : calibrate the focal length $f$, aspect ration $r$
  and the principal point $c_u$, $c_v$. This method needs at least 2
  images in which the texture model is not observed fronto-parallel to
  the 3D plane.
\item method = 5 : calibrate the focal length $f$, aspect ration $r$, skew $s$
  and the principal point $c_u$, $c_v$. This method needs at least 3
  images in which the texture model is not observed fronto-parallel to
  the 3D plane.
\end{enumerate}

Use the method number 4 when calibrating the camera for computing the odometry with \rox{}.

\item get the camera calibration:

\begin{lstlisting}
Rox_Error rox_texture_calibration_mono_perspective_get_intrinsics(Rox_Matrix intrinsics, Rox_Texture_Calibration_Mono_Perspective calibration);
\end{lstlisting}

The camera intrinsic parameters are written in a 3x3 matrix.
\end{itemize}

See the example ``rox\_example\_camera\_calibration.c'' for an example of camera calibration.

