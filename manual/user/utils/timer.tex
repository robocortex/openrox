\section{Timer}
\label{sec:timer}

The Timer module contains functions for time measurements. The module
is useful to measure the computation time when
developing.\\

\subsection{The {\tt Rox\_Timer} object}
\label{sse:timer_object}

A \lstinline$Rox_Timer$ object can be defined using the pointer to a \lstinline$Rox_Timer$:
\begin{lstlisting}
typedef struct Rox_Timer_Struct* Rox_Timer;
\end{lstlisting}

The \lstinline$Rox_Timer$ object can be used both under Linux, Windows or MacOS X without any change.  \\

\subsection{Creating/Deleting a {\tt Rox\_Timer}}
\label{sse:timer_creating}

The following function allows to create/delete a \lstinline$Rox_Timer$ object:
\begin {description}
\item [rox\_timer\_new]~: Create a \lstinline$Rox_Timer$ and allocate memory.
\item [rox\_timer\_del]~: Delete a \lstinline$Rox_Timer$ and free memory.
\end {description}

\subsection{Main functions related to {\tt Rox\_Timer}}
\label{sse:timer_methods}

The following functions shall be used for time measurement:

\begin {description}
\item [rox\_timer\_start]~: Start the timer.
\item [rox\_timer\_stop]~: Stop the timer and return the time elapsed since the initialization.
\end {description}

\noindent Example~:

\begin{lstlisting}
   //Create Rox_Timer 
   Rox_Timer timer = 0

   //Create Rox_Timer 
   rox_timer_new(&timer);

   //Start the timer
   rox_timer_start(timer);

   // Here comes the code for which you make time measurements
   ...

   // Stop the timer
   Rox_Float time_ellapsed = rox_timer_stop(timer);

   // Display time elapsed
   printf("Elapsed Time = \%f ms", time_elapsed \n);

   // Delete the timer
   rox_timer_del(&timer);
\end{lstlisting}

Information about timers is available in the Programmer Manual.
