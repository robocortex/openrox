\subsection{Matrices of the Special Euclidean Group}
\label{sse:matse3}
~\\~\\
The Special Euclidean Group SL(3) is defined by the set of 4$\times$4 matrices ${\bf T}$:
\[
{\bf T} = \left[\begin{array}{cc} {\bf R} & {\bf t} \\ 0 & 1 \end{array} \right]
\]
where ${\bf R} \in SO(3)$ is a rotation matrix that verify the following constraint:
\[
{\bf R}^\top {\bf R} = {\bf I}
\] 
Setting:
\[
{\bf R} = \left[ \begin{array}{ccc} r_{11} & r_{12} & r_{13} \\ r_{21} & r_{22} & r_{23} \\ r_{31} & r_{32} & r_{33} \end{array} \right]
\]
and
\[
{\bf t} = \left[ \begin{array}{c}  t_1 \\ t_2 \\ t_3 \end{array} \right]
\]
The matrix ${\bf T}$ can be written as:
\[
{\bf T} = \left[ \begin{array}{cccc} r_{11} & r_{12} & r_{13} & t_1 \\ r_{21} & r_{22} & r_{23} & t_2 \\ r_{31} & r_{32} & r_{33} & t_3 \\ 0 & 0 & 0 & 1 \end{array} \right]
\]

\subsubsection{The object {\tt Rox\_MatSE3}}
\label{sss:matse3}
~\\~\\
A matse3 object can be declared using the pointer to a \lstinline$Rox_MatSE3_Struct$: 

\begin{lstlisting}
typedef struct Rox_MatSE3_Struct* Rox_MatSE3;
\end{lstlisting}

The structure is opaque to the user and can only be accessed through constructors, destructors and methods described in the following sections.

\subsubsection{Creating/Deleting a {\tt Rox\_MatSE3}}
\label{sss:matse3_delnew}
~\\~\\
Functions are provided to allocate, initialize and deallocate an \lstinline$Rox_MatSE3$ object~:
\begin{lstlisting}
Rox_Error rox_matse3_new(Rox_MatSE3 * T);
\end{lstlisting}
The \lstinline$rox_matse3_new$ function allocates memory for data. By default, the matrix is initialized to the identity.\\ 

% \begin{lstlisting}
% Rox_MatSE3 rox_matse3_new_init(const Rox_Real data[12]);
% \end{lstlisting}
% The \lstinline$rox_matse3_new_init$ function first allocates memory with the default constructor and then copies the parameter `data' to initialize matrix. The data contains the entries of the matrix row by row :

% \[
% data = \left\{ \begin{array}{cccccccccccc} r_{11} & r_{12} & r_{13} & t_1 & r_{21} & r_{22} & r_{23} & t_2 & r_{31} & r_{32} & r_{33} & t_3 \end{array} \right\}
% \]

\begin{lstlisting}
Rox_Error rox_matse3_del(Rox_MatSE3 * T);
\end{lstlisting}
The \lstinline$rox_matse3_del$ function deallocates memory for an \lstinline$Rox_MatSE3$ structure. It is necessary to call this function when the
structure is not used any more or before overwriting it.

\subsubsection{Main functions related to {\tt Rox\_MatSE3}}
\label{sss:matse3_methods}
~\\~\\

Firstly, it is possible to get / set information from \lstinline$Rox_MatSE3$ structure:
\begin{description}
  \item[rox\_matse3\_get\_data\_copy]~: Get a copy of the data of the matse3 matrix.
  \item[rox\_matse3\_get\_value]~: Returns a value of the matse3 matrix.
  \item[rox\_matse3\_set\_data]~: Set the data of the matse3 matrix.
  \item[rox\_matse3\_set\_value]~: changes a value of the matrix.
\end{description}

Several functions are provided to use matrices, here the most important functions will be described to make basic matrix calculus.
\begin{description}
  \item[rox\_matse3\_mulmatmat]~: Computes the multiplication of two matrices.
  \item[rox\_matse3\_inv]~: Computes the inverse of the matrix.
  \item[rox\_matse3\_display]~: Display the matrix.
\end{description}

Other functions are provided, further information is available in the programmer manual.
