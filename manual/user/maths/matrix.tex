\subsection{Matrix}
\label{sse:matrix}

The matrix module contains matrix definitions and necessary functions to manipulate matrix.

\subsubsection{The {\tt Rox\_Matrix} object}
\label{sss:matrix_object}

A matrix object can be declared using the pointer to a \lstinline$Rox_Matrix_Struct$: 

\begin{lstlisting}
typedef struct Rox_Matrix_Struct* Rox_Matrix;
\end{lstlisting}

The structure is opaque to the user and can only be accessed through constructors, destructors and methods described in the following sections.

\subsubsection{Creating/Deleting a {\tt Rox\_Matrix}}
\label{sss:creating-deleting_matrix}
Functions are provided to allocate, initialize and deallocate an \lstinline$Rox_Matrix$ object~:
\begin{lstlisting}
Rox_Error rox_matrix_new(Rox_Matrix* matrix, const Rox_Uint cols, const Rox_Uint rows);
\end{lstlisting}
The \lstinline$rox_matrix_new$ function allocates memory for data. 
In this case, the size of allocated memory depends of parameters `cols' and `rows'.\\ 

% \begin{lstlisting}
% Rox_Matrix rox_matrix_new_init	(const Rox_Uint cols,
% 				 const Rox_Uint rows,
% 				 const Rox_Real *data);
% \end{lstlisting}
% The \lstinline$rox_matrix_new_init$ function first allocates memory with the default constructor and then copies the parameter `data' to initialize matrix.\\

% \begin{lstlisting}
% Rox_Matrix rox_matrix_new_zeros	(const Rox_Uint cols, 
% 		 		 const Rox_Uint rows);
% \end{lstlisting}
% The \lstinline$rox_matrix_new_zeros$ function first allocates memory with the default constructor and sets to 0 all matrix elements.\\

\begin{lstlisting}
Rox_Error rox_matrix_del(Rox_Matrix * M);
\end{lstlisting}
The \lstinline$rox_matrix_del$ function deallocates memory for an \lstinline$Rox_Matrix$ structure. It is necessary to call this function when the structure is not used any more or before overwriting it.

\subsubsection{Main functions related to {\tt Rox\_Matrix}}
\label{sss:matrix_function}
Several functions are provided for handling matrices. \\

Firstly, it is possible to get / set information from \lstinline$Rox_Matrix$ structure:
\begin{description}
  \item[rox\_matrix\_get\_rows]~: Returns the number of rows of the matrix.
  \item[rox\_matrix\_get\_cols]~: Returns the number of columns of the matrix.
%  \item[rox\_matrix\_get\_size]~: Returns the size of the matrix.
%  \item[rox\_matrix\_get\_data]~: Returns the data of the matrix.
  \item[rox\_matrix\_get\_value]~: Returns a value of the matrix.
  \item[rox\_matrix\_set\_value]~: changes a value of the matrix.
  \item[rox\_matrix\_set\_zero]~: Sets to 0 all matrix elements.
  \item[rox\_matrix\_set\_unit]~: Sets the identity matrix.
\end{description}

%{\bf N.B~:} The functions \lstinline$rox_matrix_set_null$ and \lstinline$rox_matrix_set_unit$ are very similar to 
%\lstinline$rox_matrix_new_null$ and \lstinline$rox_matrix_new_unit$, the main difference is that the 
%\lstinline$Rox_Matrix$ structure shall be already allocated when calling
%\lstinline$rox_matrix_set_null$ or \lstinline$rox_matrix_set_unit$.

%\paragraph{Matrix Calculus}
%\label{par:matricial_calculus}

%Many functions have been developed to use matrices, here the most important functions will be described to make basic matrix calculus.
%\begin{description}
%  \item[rox\_matrix\_add]~: Computes the addition of two matrices.
%  \item[rox\_matrix\_sub]~: Computes the subtraction of two matrices.
%  \item[rox\_matrix\_scm]~: Computes the multiplication of a matrix by a scalar.
%  \item[rox\_matrix\_mulmatmat]~: Computes the multiplication of two matrices.
%  \item[rox\_matrix\_transpose]~: Computes the transpose of a matrix.
%  \item[rox\_matrix\_inv]~: Computes the inverse of the matrix.
%\end{description}

Other functions are provided, further information is available in the programmer manual.

