\section{Image}
\label{sec:image}

The image module contains structures and methods for manipulating several image formats. 

\subsection{The object {\tt Rox\_Image}}
\label{sse:image_object}

An image object can be declared using the pointer to a \lstinline$Rox_Image_Struct$ structure. 
\begin{lstlisting}
typedef struct Rox_Image_Struct* Rox_Image;
\end{lstlisting}

\subsection{Creating/Deleting a {\tt Rox\_Image}}
\label{sse:image_newdel}

Functions are provided to allocate and deallocate a \lstinline$Rox_Image$ object~:

\begin{lstlisting}
Rox_Error rox_image_new(Rox_Image * image, Rox_Uint cols, Rox_Uint rows);
\end{lstlisting}

The function allocates memory for data depending parameters  `cols' and `rows' which correspond to image size.\\

An image can be created by directly reading a PGM file from disk using the following function:
\begin{lstlisting}
Rox_Error rox_image_new_readpgm (Rox_Image *image, const char *filename);
\end{lstlisting}
The function first allocate memory for the structure with the rows and
cols parameters matching the image file to load. Then it stores the
image data in the image object.

The following function deallocates memory for a \lstinline$Rox_Image$:
\begin{lstlisting}
Rox_Error rox_image_del(Rox_Image * image);
\end{lstlisting}
 It is necessary to call this function when the object is not used anymore.


\subsection{Main functions related to {\tt Rox\_Image}}
\label{sse:image_functs}

Once the object \lstinline$Rox_Image$ has been allocated, it is possible to get information from \lstinline$Rox_Image$.

\begin{description}
% \item [rox\_image\_get\_data]~: Get the pointer of the first pixel intensity values.
  \item [rox\_image\_get\_rows]~: Get the image rows.
  \item [rox\_image\_get\_cols]~: Get the image columns.
\end{description}

An image can be read from a PGM file if the size is compatible.
Typically, we use a \lstinline$rox_image_new_readpgm$ function to
read the first image of a sequence and allocate correctly the
corresponding structure. Then, assuming that all images of a sequence
have the same dimensions, we can re-use the same structure for the
next images, using the \lstinline$rox_image_readpgm$ function. Thus,
the structure is safely allocated while loading the reference image
and the other images are read previously to perform the tracking.
Currently, the only supported format in the library is raw PGM
(Portable Gray Map) \footnote{{http://netpbm.sourceforge.net/doc/pgm.html}}:
\begin{lstlisting}
Rox_Error rox_image_readpgm (Rox_Image image, const char *filename);
\end{lstlisting}
The function returns an error if the dimensions of the image file do not match the dimesions of the allocated image.

External libraries can be easily used to read images from files. Then,
\rox{} allows to input several image formats. The list of
available image formats is given in the enumeration
\lstinline$Rox_Image_Format_Enum$ (see the Programmer Manual for a
complete list):

\begin{lstlisting}
enum Rox_Image_Format_Enum{
     Rox_Image_Format_Grays,
     Rox_Image_Format_YUV422,
     Rox_Image_Format_Color_RGBA,
     Rox_Image_Format_Color_BGRA,
     Rox_Image_Format_Color_ARGB,
     Rox_Image_Format_Color_BGR,
     Rox_Image_Format_Color_RGB
};
\end{lstlisting}

The image data can be set using the following function:
\begin{lstlisting}
Rox_Error rox_image_set_data (Rox_Image image, Rox_Uchar *data, Rox_Uint bytesPerRow, enum Rox_Image_Format format);
\end{lstlisting}
This function transforms an external image buffer to the rox internal
format and store this information inside the Rox\_Image object.  The
input image format shall be a 3 or 4 channel interleaved image
containing Red, Blue and Green channels.  Only one of these channels
(specified by the channel parameter) will be used while converting to
rox internal format.  User needs to specify the exact format of the
input buffer (e.g. RGB, BGR, RGBA, etc.) as it is impossible to guess
it automatically, as well as how many bytes are used per row in the
input image.

Finally, in order to know if an image suitable for identification and odometry computation, the following functions allows to get a quality score.

The first function is fast but do not provide precise information: 
\begin{lstlisting}
Rox_Error rox_image_get_quality_score (Rox_Real *score, Rox_Image image);
\end{lstlisting}

The second function is slower but provides precise information on the quality of an image:
\begin{lstlisting}
Rox_Error rox_image_get_quality_score_precise (Rox_Real *score, Rox_Image image);
\end{lstlisting}
