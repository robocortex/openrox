\subsection{Texture identification}
\label{sse:ident_texture}

\noindent The texture identification module performs the identification of textured model images as shown in Figure~\ref{fig:ident_texture}. 

\begin{figure}[H]
\centering
\includegraphics[width=0.45\textwidth]{vision/figures/ident_phototexture} 
\caption{Example of image model that can be identified with the texture identification module.}
\label{fig:ident_texture}
\end{figure}

\subsubsection{The {\tt Rox\_Ident\_Texture\_SE3} object}
\label{sss:ident_texture_object}
A \lstinline$Rox_Ident_Texture_SE3$ object is a pointer to the opaque structure \lstinline$Rox_Ident_Texture_SE3_Struct$: 

\begin{lstlisting}
typedef struct Rox_Ident_Texture_SE3_Struct * Rox_Ident_Texture_SE3
\end{lstlisting}


\subsubsection{Creating/Deleting a {\tt Rox\_Ident\_Texture\_SE3}}
\label{sss:ident_texture_newdel}

\noindent Functions are provided to allocate and deallocate a \lstinline$Rox_Ident_Texture_SE3$ object~:

\begin{lstlisting}
Rox_Error rox_ident_texture_se3_new (Rox_Ident_Texture_SE3 *ident_texture);
\end{lstlisting}

\noindent The function creates a new identification structure. Shall be called before
any other function of this module. Returns a pointer to the Rox\_Ident\_Texture object. \\

\begin{lstlisting}
Rox_Error 	rox_ident_texture_se3_del (Rox_Ident_Texture_SE3 *ident_texture); 
\end{lstlisting}

\noindent Deletes an identification structure. First parameter is a pointer
created with rox\_ident\_texture\_new. Shall be called to free up memory when
user does not need identification anymore.

\subsubsection{Main functions related to {\tt Rox\_Ident\_Texture\_SE3}}
\label{sss:ident_texture_functions}
~\\

A model to be identified can be set using the following function:
\begin{lstlisting}
Rox_Error rox_ident_texture_se3_set_model (Rox_Ident_Texture_SE3 ident_texture, Rox_Model_Single_Plane model); 
\end{lstlisting}
A function is available to identify a given model in a camera object: 
The second function input the camera containing the current image in which to identify the texture:
\begin{lstlisting}
Rox_Error rox_ident_texture_se3_make (Rox_MatSE3 pose, Rox_Ident_Texture_SE3 ident_texture, Rox_Camera camera);
\end{lstlisting}

% If identification is successfull, the resulting homography matrix locating the serched texture in the image can be obtained with the following function: 
% \begin{lstlisting}
% Rox_Bool rox_ident_texture_get_matsl3_copy(Rox_MatSL3 H, Rox_Ident_Texture ident_texture)
% \end{lstlisting}

%\begin{lstlisting}
%Rox_MatSL3 rox_ident_texture_get_matsl3(Rox_Ident_Texture ident_texture) 	
%\end{lstlisting}
