\subsection{Visual Odometry Multi Plane}
\label{sse:odometry_multi_plane}

The odometry module enables tracking of a textured convex polyhedron of $n$ faces whose real
dimensions are known. The estimation of the camera pose is obtained whatever the part of the object the 
camera is looking at. The object is considered as a whole and the module uses all visible information 
to get a precise and stable localization of the object. The simplest example of a convex polyhedron 
is a cube (6 faces). The object to track using this module shall be textured on all the faces. Constraints
related to texture are similar to those of 2D planar odometry. 

\subsubsection{The {\tt Rox\_Odometry\_Multi\_Plane} object}
\label{sss:odometry_multi_plane_object}

A \lstinline$Rox_Odometry_Multi_Plane$ object is a pointer to the \lstinline$Rox_Odometry_Multi_Plane_Struct$ structure:
\begin{lstlisting}
typedef struct Rox_Odometry_Multi_Plane_Struct* Rox_Odometry_Multi_Plane;
\end{lstlisting}
The structure is opaque to the users.

\subsubsection{Creating/Deleting a {\tt Rox\_Odometry\_Multi\_Plane}}
\label{sss:odometry_multi_plane_newdel}
Functions are provided to allocate and deallocate a \lstinline$Rox_Odometry_Multi_Plane$ object~:

\begin{lstlisting}
Rox_Odometry_Multi_Plane rox_odometry_multi_plane_new(Rox_Odometry_Multi_Plane_Params params, Rox_Model_Multi_plane model);
\end{lstlisting}
The function allocates memory for the odometry object, according to the {\tt model} and {\tt params} parameters and returns a pointer on the newly created object.

A 3D object model shall be passed to the odometry module for an appropriate odometry process (see section~\ref{sec:model_multi_plane}). 
The precision of the 3D vertices passed as input is the key to the accuracy of the odometry results. 3D vertices are used 
to simulate the object image projection. If the vertices are not well defined, the alignment of the real object and the virtual 
object is not possible. Scale of the vertices is not important for the visual odometry. However the output pose varies with 
the input scale.

Carefully choosing textures is very important for a robust odometry. The picture captured for each quadrilateral shall be as much 
parallel as possible relative to the face. Blur, noise and other lighting artefacts shall be avoided. Textures shall be as big as 
possible (e.g. a 512{*}512 texture per quadrilateral). If possible, use the same device for capturing the texture and for the 
odometry process, in order to minimize differences between images.

\begin{lstlisting}
Rox_Void rox_odometry_multi_plane_del(Rox_Odometry_Multi_Plane odometry);
\end{lstlisting}
The \lstinline$rox_odometry_multi_plane_del$ function deallocates memory for a \lstinline$Rox_Odometry_Multi_Plane$ object. It is necessary to call this function when the object is not used anymore. \\

\subsubsection{Main functions related to {\tt Rox\_Odometry\_Multi\_Plane}}
\label{sss:odometry_multi_plane_methods}

The main functions to manipulate a \lstinline$Rox_Odometry_Multi_Plane$ object are~:
\begin{description}
  \item[rox\_odometry\_multi\_plane\_make]: Performs the visual odometry and update the camera pose~;
  \item[rox\_odometry\_multi\_plane\_get\_pose]: Returns the pose matrix~;
  \item[rox\_odometry\_multi\_plane\_set\_pose]: Set the pose matrix~;
  % \item[rox\_odometry\_get\_score]: Returns the quality score of visual odometry~;
\end{description}

Please refer to the Programmer Manual for further information about visual odometry functions.
See the example ``rox\_example\_odometry\_multi\_plane.c'' for an example of use.
